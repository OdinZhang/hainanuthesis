\documentclass{hainanuthesis}

\title{海南大学毕业设计模板}
\author{author_name}

\begin{document}
\begin{abstract}
    这是中文摘要
    \keywords{中文;关键字}
\end{abstract}

\begin{abstract}[en]
    This is English abstracts.
    \keywords[en]{English; Keywords}
\end{abstract}

\newpage
\tableofcontents
\newpage

\section{介绍}

本模板基于2022年海南大学毕业设计排版要求制作。
其中封面过于复杂,
暂不支持,
有需要的可以手动将其转为PDF格式之后将其导入。

\subsection{使用的宏包}

本模板目前所使用的宏包见\cref{tb:01}。

\begin{table}[!htbp]
    \begin{center}
        \caption{宏包及功能}
        \label{tb:01}
        \begin{tabular}{cc}
            \verb|fontspec| & 英文字体设置             \\
            \verb|expl3|    & \LaTeX 3语法设置         \\
            \verb|hyperref| & 启用文章超链接及PDF书签  \\
            \verb|geometry| & 页面布局设置             \\
            \verb|fancyhdr| & 页眉页脚设置             \\
            \verb|titletoc| & 目录格式设置             \\
            \verb|cleveref| & 公式、图片、表格等的引用 \\
            \verb|caption|  & 公式、图片、表格等的标号 \\
            \verb|biblatex| & 参考文献的引用           \\
        \end{tabular}
    \end{center}
\end{table}

\subsection{编译}

\begin{verbatim}
latexmk
    -synctex=1
    -interaction=nonstopmode
    -file-line-error
    -pdf
    -xelatex
    main.tex
\end{verbatim}

\section{部分命令演示}

\subsection{摘要与关键字}

\subsubsection{摘要}
\begin{verbatim}
\begin{abstract}

\end{abstract}
\end{verbatim}

上述命令将显示中文摘要。

\begin{verbatim}
\begin{abstract}[en]

\end{abstract}
\end{verbatim}

上述命令将显示英文摘要。
其中中括号中内容不为\verb|zh|时均为英文。

\subsubsection{关键字}

中文关键字为\verb|\keywords{}|,
英文关键字为\verb|\keywords[en]{}|。
同样的,
中括号内参数不为\verb|zh|时均为英文。

\subsection{致谢}

\begin{verbatim}
\begin{acknowledge}

\end{acknowledge}
\end{verbatim}

\subsection{参考文献}

在文件头使用命令
\verb|\addbibresource{filename.bib}|
引入\verb|biblatex|文件,
之后在需要输出参考文献的地方使用
\verb|\printbib|即可。

\subsection{附件}

使用\verb|\addons|命令。

\begin{acknowledge}
    致谢部分
\end{acknowledge}

\addons

这里是附件部分。
\end{document}